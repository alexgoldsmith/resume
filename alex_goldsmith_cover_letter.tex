\documentclass[a4paper, 10pt]{letter}

% Name of sender
\name{Alex Goldsmith}

% Signature of sender
\signature{Alex Goldsmith}

% Address of sender
\address{Alex Goldsmith, \\
gold.alex.smith@gmail.com\\
}

	%-----------------------------------------------------------------------------%

	\begin{document}

	% Name and address of receiver
	\begin{letter}
	{
	    Jay Painter,\\
		findhelp
	}

		    % Opening statement
		    \opening{Dear Jay,}

		    % Letter body
			I am applying for the position of Software Engineer (remote position). I am excited to read the qualifications and description of the role, as it seems a great opportunity to use my expertise for social good. The listed characteristics of the desired applicant describe me well. I am well versed in setting and meeting commitments and expectations. I am experienced in iterative workflows to improve features and usability. I have been a mentee and a mentor in a professional setting. The ability to communicate requirements and solutions for audiences of disparate technical literacy was essential to my role at Epic.

			Regarding the technical requirements, I have experience writing Python for a variety of uses, including application development, web scraping, and data science. I have used the React framework to develop web applications (and dabbled in Vue and Svelte). I have worked with both relational and non-relational databases, and am proficient in SQL. I am familiar with web performance strategies and metrics (I am currently obsessed with the JAMstack to maximize first-load performance). GCP currently hosts virtual machines I have designed for my personal projects, and I have gathered decent knowledge on the platform in the process. Microservices, specifically cloud-hosted serveless functions and 3rd party APIs are key to developing with my preferred web architecture, the JAMstack. 

			I am interested in working specifically at findhelp because I believe findhelp's mission aligns with my own personal mission of using technology to advance social good. While working at Epic, I learned about the incredible impact of social determinants of health on personal care and general qualitify of life. The application I supported included functionality for financial assistance workers, and I greatly valued the times I was able to aid their work. The characteristic I value most in the firm I work for next is a commitment to social good above the pursuit of profit. I believe findhelp may be that workplace.    

			I should also give a brief overview of my professional and academic experience. From Spring 2021 through Summer 2022, I completed coursework in computer science at the University of Wisconsin-Madison. In the Summer of 2022, I developed a demo web application for a course project using PyScript, JavaScript, and SQLite. The Web application allowed users to query a database, and subsequently displayed a visualization of the returned results. In Fall of 2021, I worked in a three-person team to develop a mental-health cross-platform application. I contributed to both front-end (React-native) and back-end (NodeJS) development. I also wrote all automated tests for the application using Jest. Previous coursework at UW-Madison focused on data structures and algorithms. 

			From December 2020 to June 2021, I worked as a Technical Solutions Engineer at Epic Systems Corporation, where I supported six Healthcare Systems in the maintenance phase of the software development lifecycle, as well as one customer undergoing implementation. During that time, I worked remotely with IT teams all over the country to set priorities, troubleshoot issues, and discover creative solutions to various technical and business problems. This experience led me to pursue a more development focused career path, which is why I enrolled at UW-Madison.

			Prior to my tenure at Epic Systems Corporation, I graduated with a Masters of Economics from Clemson University. I used Python to process, analyze, and visualize data for my Master’s Thesis. Specifically, I wrote Python scripts to load data from a number of flat files, used the scikit-learn library to construct a number of analytical models, and created visualizations of the results using Matplotlib.

			I would be glad to speak further about how I may be an asset at findhelp. If you have any questions for me, please reach out over email at gold.alex.smith@gmail.com. 
		    % Closing statement
		    \closing{Best,}
		    \end{letter}
		    \end{document}
		    %-----------------------------------------------------------------------------%
